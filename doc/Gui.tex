\chapter{Implementing the GUI}

I've chosen the \emph{Simple and Fast Multimedia Library}
\href{https://www.sfml-dev.org/index.php}{SFML} for creating the graphical user interface
of the app.
It lives up to its name and is available on all major platforms and programming languages.

In this chapter we'll learn how to display the board and the chess pieces on it.
In order to show valid moves for each piece, we will also learn the basic rules of the game.

I'm using \href{https://cmake.org/}{CMake} for building the C\textsuperscript{++} code,
and the SFML CMake project template, which will build the SFML libraries.
So you will need to have the following components installed on your machine:

\begin{itemize}
  \item a decent C\textsuperscript{++} compiler (any of the major compilers will do)
  \item the \emph{git} tool
  \item the \emph{cmake} tool
  \item the required system packages for
    \href{https://www.sfml-dev.org/tutorials/2.6/start-cmake.php}{SFML}
\end{itemize}
 
On a linux system, all those components can be installed with your systems package manager
(e.g. with \texttt{apt-get} on Ubuntu).
The same is true for Mac OS, just use the included \texttt{clang++} compiler and install
missing components with \href{https://brew.sh/}{homebrew}:
\texttt{<brew install git cmake sfml>}.

When everything is in place, just clone my repository with e.g.\\
\texttt{<git clone https://github.com/okrischer/ThinkChess.git>} and execute\\
\texttt{<cmake -B build>} from the root folder of your local copy.\\
If everything went well, change to the \texttt{build} folder and execute
\texttt{<cmake --build .>} and voila, you have a running app, which you can start with
\texttt{<./ThinkChess>}.

\section{Displaying the board and the pieces}


\section{Showing valid moves}

